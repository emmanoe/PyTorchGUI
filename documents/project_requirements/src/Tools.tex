To lead this project to the end, we are going to use some tools that will help us accomplish this project.

\section{PyTorch library}

PyTorch is an open-source machine learning library for Python. It is based on Torch, which is an old machine learning library (first release : October 2002), itself based on the Lua programming language. PyTorch library allow users to be able to create neural networks, edit existing models, and make calculations on those networks with high-level abstraction functions.

A neural network is a collection of neurons interconnected and permits the resolution of complex problems like graphical recognition or the natural language treatment by the blessing of the  of the weighting adjustment coefficients in a learning phase.
The idea of the program is to read a file (list in Elicitation Requirements), choose an algorithm, edit its parameters then perform a calculation thanks to PyTorch then it returns different results depending the different parameters you entered previously. Results can be found in the form of pictures edited from the original, graphs and statistics data.
\newline This application can be used by anyone wishing to use the PyTorch Deep Learning Library experiment and create models. Users can be scientist, students, everyone that is interested in Deep Learning practice

\section{HiddenLayer library}
HiddenLayer is a A lightweight library for neural network graphs and training metrics for PyTorch, Tensorflow, and Keras developed by Waleed Abdulla  and Phil Ferriere, and is licensed under the MIT License.
\newline We choose to use this library to render graphs of neural networks and export them as pdf or png files.

\section{Jupyter NoteBook}
HiddenLayer work with jupyter notebook. Jupyter Notebook is a great tool for data scientist. We will use it in background of our applciation for graph computing data representation and training metric. It provide us, a great background environment. 
\section{Integrated development environment}
We all have our familiarities with some integrated development environment (IDE).
This are those we will most likely be using : 
    - Emacs
    - Atom
    - Sublime Text
    - Visual Studio Code
    - Qt Creator

\section{Libraries}
We will use some libraries that will ease us the task. Those libraries might change over the progress of our project :
\begin{itemize}
    \item PyTorch 
    \item Qt: Since we plan on using QT Creator to create our graphical interface, we will surely make use of the QT Library, which provides a lot of graphical elements.
    \item Plotly: Using this library could be very useful. It can indeed provide a lot of graphical features, like plotting data, and generating graphs.
    \item Numpy: Numpy is an extension to the Python language, which allows to manipulate matrices and multidimensional arrays, and to use a variety of mathematical operations on it. Numpy is already integrated to PyTorch.
    \item Networkx: A python library for studying graphs and networks.
    \item h5py: Our application needs to be able to read hdf5 files. This library allows to read them.
\end{itemize}

\section{Utilities}

\begin{itemize}
    \item Version control system (VCS)
        As we work as a group and not individually on totally independent tasks, we have to use a VCS. Since we all know GitHub, we chose to use this simple VCS.
\end{itemize}
